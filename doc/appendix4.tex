% This file was created by the WP2LaTeX program version: 3.51 
\documentclass[11pt]{article}
\usepackage{wp2latex}
\usepackage[USenglish]{babel}


\begin{document}
\setcounter{page}{127}\pagenumpos{\pnbr}
\section{  APPENDIX 4  }

\bigskip
{\bf {\large Establishing the sowing date }}

\bigskip
In the chapters four to six the simulation processes are being explained. As stated in
\S 5.2.1, the start of the growing season can be either the date of sowing or the date of
emergence. If the sowing date is not given by the user, it will be deter\-mined by the
model. This is done in subroutine {\bf STDAY}.

\bigskip
\bigskip
First the workability of the soil over the period before sowing is checked on the basis of
soil and crop specific data. The workability of the soil is dependent of the excess amount
of water in the plow layer. It is assumed that in case the excess amount of water during
the previous time step is more than 0.5 cm, the capillary rise to surface is negligible\\
and the soil evaporation equals the evaporation of a wet surface calculated with {\bf PEN\-MAN} (see \S 4.1.1). If the excess amount of water in the plow layer is less than 0.5 cm,
the maximum capillary rise has to be considered. In subroutine {\bf STDAY} the maxi\-mum
capillary rise is calculated with an AFGEN table (acronym: {\bf CAPFRU}) using the excess
amount of water in the plow layer as an independent variable. This AFGEN table is soil
dependent, however, in the model only one fixed table is used. The evaporation from the
soil can be calculated by:

\begin{displaymath}
E _{s} ~= ~\min \, (EO _{s} \, ,\, CR _{\max } ~+~ P)
\end{displaymath}

\bigskip
\nwln
\begin{tabbing}
\hspace{1.27cm}\=\hspace{1.27cm}\=\hspace{1.27cm}\=\hspace{1.27cm}\=%
\hspace{1.27cm}\=\hspace{1.27cm}\=\hspace{1.27cm}\=\hspace{1.27cm}\=%
\hspace{1.27cm}\=\hspace{1.27cm}\=\kill
where\> E$_{{\rm s}}$\> : Evaporation rate from a bare soil\> \> \> \> \> \> \> \> [cm d$^{{\rm -1}}$]\\
\>EO$_{{\rm s}}$\> : Potential evaporation rate from a bare soil \\
\>\>   surface (see \S 4.1.3)\> \> \> \> \> \> \> \> [cm]\\
\>CR$_{{\rm max}}$\> : Maximum capillary rise\> \> \> \> \> \> \> \> [cm d$^{{\rm -1}}$]\\
\>P \> : Precipitation intensity\> \> \> \> \> \> \> \> [cm d$^{{\rm -1}}$]
\end{tabbing}

\bigskip
\bigskip
The excess amount of water in the plow layer at time step t can now be established
according to:

\begin{displaymath}
W _{exc,t} ~= ~\max \, (-1.\, ,\, W _{exc,t-1} ~+~ P ~+~E _{s} )\,\Delta t
\end{displaymath}

\bigskip
\nwln
\begin{tabbing}
\hspace{1.27cm}\=\hspace{1.27cm}\=\hspace{1.27cm}\=\hspace{1.27cm}\=%
\hspace{1.27cm}\=\hspace{1.27cm}\=\hspace{1.27cm}\=\hspace{1.27cm}\=%
\hspace{1.27cm}\=\hspace{1.27cm}\=\kill
where\> W$_{{\rm exc,t}}$\> : Excess amount of water in the plow layer \\
\>\>   at time step t\> \> \> \> \> \> \> \> [cm]\\
\>E$_{{\rm s}}$\> : Evaporation rate from a bare soil\> \> \> \> \> \> \> \> [cm d$^{{\rm -1}}$]\\
\>P \> : Precipitation intensity\> \> \> \> \> \> \> \> [cm d$^{{\rm -1}}$]
\end{tabbing}

\bigskip
\bigskip
A correction has to be made for the infiltration during the present time step in case the
excess amount of water in the plow zone is more than zero. The infiltrating amount of
water of water can be calculated by:

\begin{displaymath}
\IN _{pl} ~=~\min \, (W _{ext\, ,\, t} ~ Sp _{1} ~+~Sp _{2} \, ,\, W _{ext\, ,\, t} \, )
\end{displaymath}

\bigskip
\nwln
\begin{tabbing}
\hspace{1.27cm}\=\hspace{1.27cm}\=\hspace{1.27cm}\=\hspace{1.27cm}\=%
\hspace{1.27cm}\=\hspace{1.27cm}\=\hspace{1.27cm}\=\hspace{1.27cm}\=%
\hspace{1.27cm}\=\hspace{1.27cm}\=\kill
where\> In$_{{\rm pl}}$\> : Amount of water infiltrating in the plow layer\> \> \> \> \> \> \> \> [cm]\\
\>W$_{{\rm exc,t}}$\> : Excess amount of water in the plow layer \\
\>\>   at time step t\> \> \> \> \> \> \> \> [cm]\\
\>Sp$_{{\rm 1}}$\> : First top soil seepage parameter\> \> \> \> \> \> \> \> [-]\\
\>Sp$_{{\rm 2}}$\> : Second top soil seepage parameter\> \> \> \> \> \> \> \> [-]
\end{tabbing}

\bigskip
\bigskip
The soil is considered workable when the excess amount of water in the plow layer is
below a preset value. This value is soil specific (acronym: {\bf DEFLIM}) and should be
provided by the user. The number of workable days is established. However, in case a
certain day cannot be considered as a workable day, the sum of workable days is set back
to zero. It should be mentioned that there exist two types of seepage parameters. One for
shallow seedbeds and another for deep seedbeds. The user should indicate which one
should be used.

\bigskip
The sowing can take place when more then three consecutive workable days occur and if
the sowing date falls within a predefined period, the earliest sowing date (acronym:
{\bf IDESOW}) and the latest sowing date (acronym: {\bf IDLSOW}). This period is crop specific
and should be provided by the user. 
\end{document}
