% This file was created by the WP2LaTeX program version: 3.51 
\documentclass[11pt]{article}
\usepackage{wp2latex}
\usepackage{graphicx}

\graphicspath{{./}}

\begin{document}
\setcounter{page}{122}\pagenumpos{\pnbr}
\section{  APPENDIX 2  } 

\bigskip
{\bf {\large AFGEN function}}

{\bf AFGEN} stands for {\bf A}rbitrary {\bf F}unction {\bf GEN}erator. It is a fortran function which is used for
linear interpolation in a one-dimensional array with paired data. The uneven places in the
array represent the X-values, whereas the Y-values are represented by the even places of the
array. Such an array can be used to describe the dependency of variable Y of variable X in
case no mathematical description is available or is too cumbersome. A plotted example is
depicted in figure A2:

\begin{figure}[htbp]
 \begin{center}\includegraphics[width=10.05cm]{AFGEN.eps} \end{center}
\end{figure}

\bigskip
\bigskip
\bigskip
\bigskip
\bigskip
\bigskip
\bigskip
\bigskip
\bigskip
\bigskip
\bigskip
\bigskip
\bigskip
\bigskip
\bigskip
\bigskip
\nwln
\begin{center}
Fig. A2 Linear interpolation
\end{center}

 \bigskip
The array belonging to this example has to be filled as: \nwln
\begin{tabbing}
\hspace{1.27cm}\=\hspace{1.27cm}\=\hspace{1.27cm}\=\hspace{1.27cm}\=%
\hspace{1.27cm}\=\hspace{1.27cm}\=\hspace{1.27cm}\=\hspace{1.27cm}\=%
\hspace{1.27cm}\=\hspace{1.27cm}\=\kill
\>\> Place\> (1)\> (2)\> (3)\> (4)\> (5)\> (6)\> (7)\> (8)\\
\>\> Value\> X$_{{\rm 1}}$ \> Y$_{{\rm 1}}$   \> X$_{{\rm 2}}$\> Y$_{{\rm 2}}$   \> X$_{{\rm 3}}$ \> Y$_{{\rm 3}}$   \> X$_{{\rm 4}}$ \> Y$_{{\rm 4}}$
\end{tabbing}

 \bigskip
The arguments of the AFGEN function in order of their place in the argument list are: name
of the table, number of pairs, X value to be interpolated. The X-values have to be arranged
from low to high values and are not allowed to be interchanged. Every X-value has to
precede its connected Y-value.

\bigskip
\bigskip
\bigskip
\bigskip
\bigskip
\bigskip
Three situations for interpolation can occur:

1)
\testlastline

\begin{indenting}{1.27cm}
The argument {\it x\/} at which interpolation should take place is less or equal to the first
X-value in the array. The Y-value is set to the first Y-value in the array.
\end{indenting}

 \bigskip
2)
\testlastline

\begin{indenting}{1.27cm}
The argument {\it x\/} value at which interpolation should take place is between the first
and the last X-value in the array. The Y-value can now be found via linear
interpolation. First the X values left and right from the argument {\it x\/} have to be
detected, then the Y-value can be calculated:
\end{indenting}

\begin{displaymath}
y~=~ Y _{n-1} ~+~ (\, x\, -\, X _{n-1} \, )\,{\frac{ Y _{n} \, -\, Y _{n-1} }{X _{n} \, -\, X _{n-1} }}
\end{displaymath}

\bigskip
\bigskip
\bigskip
\bigskip
3)
\testlastline

\begin{indenting}{1.27cm}
The argument {\it x\/} at which interpolation should take place is equal to or larger then the
last X-value in the array. The Y-value is set to the last Y-value in the array.
\end{indenting}
\end{document}
